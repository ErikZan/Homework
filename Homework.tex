\documentclass[onecolumn]{article}
\usepackage[utf8]{inputenc}
\usepackage{amsmath} % Advanced math typesetting
\usepackage[utf8]{inputenc} % Unicode support (Umlauts etc.)
\usepackage[english]{babel} % Change hyphenation rules
\usepackage{hyperref} % Add a link to your document
\usepackage{graphicx} % Add pictures to your document
\usepackage{listings} % Source code formatting and highlighting
\usepackage{bookmark}
\usepackage{natbib}
\usepackage{geometry}
\usepackage{bbm}
%\usepackage{multicol}
\usepackage{fancyhdr}
\usepackage[document]{ragged2e}
\usepackage{adjustbox}
\usepackage{subcaption}
% \pagestyle{fancy}
% \fancyhf{}
% \fancyhead[LE,RO]{Overleaf}
% \fancyhead[RE,LO]{Guides and tutorials}
% \fancyfoot[CE,CO]{\leftmark}
% \fancyfoot[LE,RO]{\thepage}
\graphicspath{ {Immagini/} }
%C:\Users\erikz\OneDrive\Desktop\PaioPaio2\IMM-Sensors-Network\Latex\Immagini
\geometry{
 a4paper,
 total={170mm,257mm},
 left=20mm,
 top=20mm,
 }

\title{Homework AORC}
\author{
Pasquale Romano 198573 

Zanolli Erik 198852}
\date{April 2020}





\begin{document}

\maketitle



\section*{1}
\justify
The strategy that leads to the smallest tracking error is the $tanh(2*dq)$ method . The MSE value referred to various $Kp Kd Ki$ are listed in Table below

The reason is the fact that the $tanh(2*dq)$ can combine the property of continuity and a good approximation of the step function that characterize the model of Coulomb friction.
 But that dependes on the range of frequency and velocity that we are working on: when velocitys are small and they remains near zero for "long" time we have an incorrect estimation 
 of our Coulomb physical model and so the $tanh(2*dq)$ method generate and higher Tracking Error. In this situation the smallest Tracking error is generated by the sign function

\section*{2}
The strategy that leads to the highest tracking error is the $tanh(0.01*dq)$ method. The reason is probably that it approximate badly the fisical Coulomb model inducing error in the 
vicinity of $\dot{q}=0$ at a more higher rate of the $tanh(2*dq)$ method. 

\section*{3}
the main problem is that the sign is a discontinous function so in the proximity of zero it have a infinite derivative. close to this points an small error in the retrive the position/velocity can lead to 
instantaneous change in the torque value applied and generate oscillation and vibration.

\section*{4}
First of all we have to consider the range of velocity that we are going to use. a robot that is going to use small torque and velocity can work better with a sign method
even if it has a discontinuity ... Maybe

\end{document}